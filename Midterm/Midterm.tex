\documentclass[12pt,a4paper]{article}
\usepackage[utf8]{inputenc}
\usepackage[T1]{fontenc}

\usepackage{graphics,amsmath,amsfonts,amsbsy,amssymb,amsthm}
\usepackage[a4paper]{geometry}
\usepackage{graphicx}
\usepackage{hyperref}
\usepackage{datatool}
\usepackage{float}
\usepackage{framed}
\usepackage{listingsutf8}
\usepackage{enumerate}
\usepackage{comment}
\usepackage{epstopdf}
\usepackage{cancel}
\usepackage{caption}
\usepackage{subcaption}
\usepackage{tikz}
\usepackage{mathtools}
%\usepackage{algorithmicx}
\begin{document}

\section*{Lines}

\subsection*{Equation of a line}

$(4, -3)$ and $(-1, 1)$
\newline

\noindent $y = mx + b$

\noindent $ y - y_{0} = m(x - x_{0})$
\newline

\subsubsection*{Slope of line}

\noindent $(-1 - (-3)) = m(1 - 4)$

\noindent $m = \frac{(1 - (-3))}{-1 - 4}$

\noindent $\boxed{m = -\frac{4}{5}}$
\newline


\subsubsection*{Point Slope}

\noindent $(y - (-3)) = \frac{-4}{5}(x - 4)$

\noindent $y + 3 = -\frac{4}{5}(x - 4)$

\noindent $y + 3 = -\frac{4}{5}x + \frac{16}{5}$

\noindent $y= -\frac{4}{5}x + \frac{16}{5} - 3$

\noindent $y= -\frac{4}{5}x + \frac{16}{5} - \frac{15}{5}$

\noindent $\boxed{y= -\frac{4}{5}x + \frac{1}{5}}$

\subsubsection*{Slope intercept}

\noindent $y = mx + b$

\noindent $-3 = (-\frac{4}{5} \times 4) + b$

\noindent $-3 = -\frac{16}{5} + b$

\noindent $b = -3 - (-\frac{16}{5})$

\noindent $b = -\frac{15}{5} + \frac{16}{5}$

\noindent $\boxed{b = \frac{1}{5}}$

\newpage

\section*{Angles}


\noindent $cos \frac{\pi}{6} = \frac{\sqrt{3}}{2}$

\noindent $cos \frac{\pi}{4} = \frac{\sqrt{2}}{2}$

\noindent $cos \frac{\pi}{3} = \frac{1}{2}$

\noindent $cos \frac{\pi}{2} = 1$
\newline

\noindent It's not possible to find exact value of $cos \frac{7\pi}{12}$

\noindent Then we need to use:

\begin{itemize}
    \item $cos(\pi - x) = -cos x$
    \item $cos(\frac{\pi}{2} - x) = sin x$
    \item $cos(x + y) = cos(x)cos(y) - sin(x)sin(y)$
\end{itemize}


\noindent We can represet $\frac{7\pi}{12}$ as a sum: $\frac{3\pi}{12} + \frac{4\pi}{12} = \frac{\pi}{4} + \frac{\pi}{3}$

\noindent $cos \frac{7}{12} = cos (\frac{\pi}{4} + \frac{\pi}{3}) = cos(\frac{\pi}{4})cos(\frac{\pi}{3}) - sin(\frac{\pi}{4})sin(\frac{\pi}{3})$

\noindent $ = \frac{\sqrt{2}}{2}cos(\frac{\pi}{3}) - \frac{\sqrt{2}}{2}sin(\frac{\pi}{3})$

\noindent $ = \frac{\sqrt{2}}{4} - \frac{\sqrt{2}\sqrt{3}}{4}$

\noindent $ = \boxed{\frac{\sqrt{2} - \sqrt{6}}{4}}$


\newpage


\section*{Simplify}

$(\frac{3y\sqrt{x}}{6x^3 y^{-4}})^2$

\noindent $(\frac{y\sqrt{x}}{3x^3 y^{-4}})^2$


\noindent $= ((y x^{\frac{1}{2}})(2x^{3} y^{-4})^{-1})^2$

\noindent $= ((y x^{\frac{1}{2}})(\frac{1}{2}x^{-3} y^{4}))^2$

\noindent $= (yx^{\frac{1}{2}} \frac{1}{2}x^{-3} y^4)^2$

\noindent $= (\frac{1}{2}x^{\frac{1}{2} + (-3)} y^{1+4})^2$

\noindent $= (\frac{1}{2}x^{\frac{1}{2} - \frac{6}{2}} y^{1+4})^2$

\noindent $= (\frac{1}{2}x^{\frac{-5}{2}} y^5)^2$

\noindent $= (\frac{1}{4} x^{\frac{-5}{2} + \frac{-5}{2}} y^{10})$

\noindent $= (\frac{1}{4} x^{\frac{-10}{2}} y^{10})$

\noindent $= (\frac{1}{4} x^{-5} y^{10})$

\noindent $\boxed{= \frac{y^{10}}{4 x^{5}}}$


\newpage


\section*{Simplify Logarithms}


\begin{itemize}
    \item $log \frac{a}{b} = log (a) - log (b)$
    \item $log(a \times b) = log (a) + log (b)$
    \item $log(a^b) = b \times log (a)$
    \item $x^a x^b = x^{a+b}$
\end{itemize}

\subsubsection*{Example:}

Expand and simplify $log_{4} \frac{a^4 \sqrt[3]{4}}{2e^{-ln(64)}}$

\noindent $log_{4}(a^4 \sqrt[3]{4}) - log_{4}(2e^{-ln(64)})$

\noindent $= ((log_{4}(a^4))+(log_{4}(\sqrt[3]{4}))) - ((log_4(2))+(log_{4}(e^{-ln(64)})))$

\noindent $= (log_{4}(a)+(log_{4}(4^{\frac{1}{3}}))) - (log_{4}(\sqrt{4}) + ((log_4(2)) + log_{4}(64)))))$

\noindent $= (log_{4}(a^4)+(log_{4}(4)^\frac{1}{3}) - log_{4}(\sqrt{4}) - (log_{4}(2)) + ((log(64))))$

\noindent $= (log_{4}(a^4)+ log_{4}(4^{\frac{1}{3}}) -  log_{4}(2)+ (log_{4}(64))$

\noindent $= 4log_{4}(a) + \frac{1}{3} - \frac{1}{2} + 3$

\noindent $= \boxed{4log_{4}(a) + \frac{17}{6}}$



\newpage

\section*{Product Rule Practice}

\subsection*{1.}

$\frac{d}{dx} x arctan x = \frac{d}{dx} f(x) \times g(x)$
\newline

\noindent $f(x) = x$

\noindent $f'(x) = 1$

\noindent $g(x) = arctan (x)$

\noindent $g'(x) = \frac{1}{1 + x^2}$
\newline

\noindent $\frac{d}{dx} f(x)g(x) = f(x)g'(x) + f'(x)g(x)$

\noindent $\frac{d}{dx} x arctan(x) = x \times \frac{1}{1 + x^2} + arctan(x) * 1$



\subsection*{2.}

\noindent $\frac{d}{dx} \frac{ln x}{x}$
\newline

\noindent $f(x) = ln(x)$

\noindent $f'(x) = \frac{1}{x}$

\noindent $g(x) = x^-1$

\noindent $g'(x) = -x^(-2)$
\newline

\noindent $\frac{d}{dx} = ln(x) \times (-x^{-2}) + \frac{1}{x} \times x^{-1} = \frac{-ln(x)}{x^2} + \frac{1}{x^2} = \frac{1 - ln(x)}{x^2}$

\subsection*{3.}

\noindent $\frac{d}{dx} 4x^5 \sqrt{x}$
\newline

\noindent $f(x) = 4x^5$

\noindent $f'(x) = 5 \times 4 x^{5-1} = 20 x^4$

\noindent $g(x) = \sqrt{x}$

\noindent $g'(x) = \frac{1}{2\sqrt{x}}$
\newline

\noindent $\frac{d}{dx} 4x^4 \sqrt{x} = 4x^2 \times \frac{1}{2\sqrt{x}} + 20x^4 \times \sqrt{x} = 2x^{\frac{9}{2}} + 20x^{\frac{9}{2}} = 22x^{\frac{9}{2}}$

\newpage

\section*{Chain Rule}

\noindent $\frac{d}{dx} f(g(x)) = g'(x) \times f'(g(x))$

\subsection*{1.}
\noindent $\frac{d}{dx} e^{sin(x)+x^2}: f(x) = e^{g(x)}$
\newline

\noindent $f(x) = e^x$

\noindent $f'(x) = e^x$

\noindent $g(x) = sin(x) + x^2$

\noindent $g'(x) = cos(x) + 2x$
\newline

\noindent $\frac{d}{dx} sin(x) + x^2 = (cos(x) + 2x)e^{sin(x)+x^2}$

\subsection*{2.}
\noindent $\frac{d}{dx} \frac{1}{2+3x^4}: f(x) = \frac{1}{g(x)}$
\newline

\noindent $f(x) = \frac{1}{x}$

\noindent $f'(x) = \frac{-1}{x^2}$

\noindent $g(x) = 2 + 3x^2$

\noindent $g'(x) = 6x$
\newline

\noindent $\frac{d}{dx} \frac{1}{2+3x^4} = 6x\frac{-1}{(2+3x^4)^2} = \frac{-6x}{(2+3x^4)^2}$

\subsection*{3.}
\noindent $\frac{d}{dx} 2cos^3 (x): f(x) = 2(g(x))^3$
\newline

\noindent $f(x) = 2x^3$

\noindent $f'(x) = 6x^2$

\noindent $g(x) = cos(x)$

\noindent $g(x) = -sin(x)$
\newline

\noindent $\frac{d}{dx} 2cos^3 (x) = (-sin(x))(6cos(x)^2) = -6(sin(x)(cos(x))^2$

\newpage

\section*{All Local Minima and All Local Maxima}
Occurs at critical points so we need f'(x).

\subsection*{1.}
\noindent $g(x) = xe^{-x^2}$

\noindent $g'(x) = -2xe^{-x^2}$

\noindent $h(x) = -x^2$

\noindent $h'(x) = 2 \times -x^{2-1} = -2x$
\newline

\noindent $f'(x) = h'(x) \times g(x) + h(x) \times g'(x)$

\noindent $f'(x) = (1) \times (e^{-x^2}) + (x) \times (-2xe^{-x^2})$

\noindent $f'(x) = ((e^{-x^2}) + (-2x^2 e^{-x^2}))$

\noindent $\boxed{f'(x) = (1 - 2x^2)e^{-x^2}}$

$P_1 = \frac{1}{\sqrt{2}}$

$P_2 = -\frac{1}{\sqrt{2}}$

\subsubsection*{2.}

since $e^x > 0$  the $(1 - 2x^3)$ is the critical point.


\noindent $g(x) = (1 - 2x^2)$

\noindent $g'(x) = -4x$

\noindent $f''(x) = g'(x)e^{-2x} + g(x)g'()$

$= (-4x)e^{-x^2} + (1 - 2x^2)(-2xe^{-x^2})$

$= (-6x+4x^3)e^{-x^2}$



$P_1 = \frac{1}{\sqrt{2}}$ (local maxmimum)

$P_2 = -\frac{1}{\sqrt{2}}$ (local mimiumum)

\newpage

\section*{Global Optimization}

$f(x) = arctan(x) - \frac{x}{2}$ at interval $[-\sqrt{3}, \frac{1}{\sqrt{3}}]$

\noindent $f'(x) = \frac{1}{1 + x^2} - \frac{1}{2}$

\noindent Zeroes at $x = -1$ and $\cancel{x = 1}$

\noindent $f(-\sqrt{3}) = arctan(-\sqrt{3}) - \frac{-\sqrt{3}}{2} = -\frac{\pi}{3} - \frac{-\sqrt{3}}{2} = \frac{3\sqrt{3} - 2\pi}{6}$

\noindent $f(-1) = arctan (-1) - \frac{-1}{2} = -\frac{\pi}{4} - \frac{-1}{2} = \frac{2-\pi}{4}$

\noindent $f(\frac{1}{\sqrt{3}}) = arctan (\frac{1}{\sqrt{3}}) - \frac{\frac{1}{\sqrt{3}}}{2} = \frac{\pi}{6} - \frac{1}{2\sqrt{3}} = \frac{\pi - \sqrt{3}}{6}$

\noindent $f(x)$ acheves a maxmimum value of $\frac{\pi - \sqrt{3}}{6} = \frac{1}{\sqrt{3}}$

\noindent $f(x)$ acheves a minimum value of $\frac{2 -\pi}{4}$ at $x=-1$.

\newpage

\section*{Taylor Approximation}

\noindent find the second order Taylor approximation of $f(x) = arctan(\sqrt{x})$ around the point $a=1$.

\subsection*{f'}

\noindent Let $g_1(x) = \sqrt{x}$

\noindent $g_1'(x) = \frac{1}{2\sqrt{x}}$

\noindent Then $f_1'(x) = \frac{1}{1+g(x)^2}g_1'(x)$

\noindent $= \frac{1}{1+\sqrt{x^2}} \times \frac{1}{2\sqrt{x}} = \frac{1}{2\sqrt{x} + 2x^{\frac{3}{2}}}$

\subsection*{f''}

\noindent Let  $g_2(x) = 2\sqrt{x} + 2x^{\frac{3}{2}}$

\noindent $g_2'(x) = \frac{1}{\sqrt{x}} + 3\sqrt{x}$.

\noindent Then $f''(x) = -1 \times g_2(x)^{-2} \times g_2'(x)$

$=[-(2\sqrt{x} + 2x^{\frac{3}{2}})^{-2}] \times (\frac{1}{\sqrt{x}} + 3\sqrt{x})$

\subsection*{Solution}

\noindent $f(1) = arctan(\sqrt{1}) = arctan (1) = \frac{\pi}{4}$

\noindent $f'(1) = \frac{1}{2\sqrt{1} + 2(1)^{\frac{3}{2}}} = \frac{1}{2+2} \frac{1}{4}$

\noindent $f''(1) = -(2\sqrt{1} + 2(1)^{\frac{3}{2}})^{-2}(\frac{1}{\sqrt{1}} + 3\sqrt{1}) = -4^{-2} 4 = -\frac{1}{4}$

\noindent $f(1) + \frac{f'(1)}{1!}(x-1) + \frac{f''(1)}{2!}(x-1)^2 = \frac{\pi}{4} + \frac{\frac{1}{4}}{1!} + \frac{\frac{-1}{4}}{2!}(x-1)^2$

\newpage

\section*{Optimization word problem}

\noindent we have $45 m^2$ of material to build a box with a square base and no top. Determine the dimensions of the box that will maximize the enclosed volume.
\newline

\noindent volume formulas

\noindent $V = l \times w \times h$

\noindent $V = w^2 \times h$
\newline

\noindent $V = w^2 (\frac{45 - w^2}{4w})$

\noindent $V = \frac{1}{4} (45w-w3)$
\newline

\noindent material used:

\noindent $M = 1 * w^2 + 4 \times wh$
\newline

\noindent $45 = w^2 + 4wh$

\noindent $h = \frac{45 - w^2}{4w}$
\newline

\noindent our goal is to maximize V, so we differentiate to find critical points

\noindent $V' = \frac{1}{4} (45 - 3w^2)$

\noindent Zeroes at: $w = \sqrt{15}$ and \cancel{$w = -\sqrt{15}$}

\noindent The first derivative test tells us that $\sqrt{15}$ is a maximum.

\subsection*{Solution}

we need to solve for h.

$h = \frac{45 - w^2}{4w}$

$h = \frac{45-\sqrt{15}^2}{4\sqrt{15}}$

$h = \frac{45 - 15}{4\sqrt{15}}$

$h = \frac{15}{2\sqrt{15} = \frac{\sqrt{15}}{2}}$

Width of $\sqrt{15}$.

Length of $\sqrt{15}$.

Height of $\sqrt{15}$.

\newpage

\section*{polar to cart}
To convert the polar coordinate (4, π/3) to Cartesian coordinates: 
$x = 4 * cos(π/3) = 4 * (1/2) = 2$
$y = 4 * sin(π/3) = 4 * (√3/2) = 2√3$


\end{document}

