\documentclass[12pt,a4paper]{article}
\usepackage[utf8]{inputenc}
\usepackage[T1]{fontenc}

\usepackage{graphics,amsmath,amsfonts,amsbsy,amssymb,amsthm}
\usepackage[a4paper]{geometry}
\usepackage{graphicx}
\usepackage{hyperref}
\usepackage{datatool}
\usepackage{float}
\usepackage{framed}
\usepackage{listingsutf8}
\usepackage{dsfont}
\usepackage{enumerate}
\usepackage{comment}
\usepackage{epstopdf}
\usepackage{caption}
\usepackage{subcaption}
\usepackage{tikz}
\usepackage{mathtools}
%\usepackage{algorithmicx}
\begin{document}

\subsection*{Derivatives}

$f(x)= sin(x)$
$f'(x)= cos(x)$
$f''(x)= -sin(x)$
$f'''(x)= -cos(x)$


$f(x) = 2sin(x) + 3cos(x)$
$f(x) = ag(x) + bh(x)$
$f'(x) = aq'(x) + bh'(x)$

$f(x) = e^x$
$f'(x) = e^x$

$f(x) = e^(2sin(x) + 4cos(x))$
$f(x) = e^x$
$f'(x) = e^x$

$g(x) = 2sin(x) + 4cos(x)$
$g'(x) = 2cos(x) - 4sin(x)$


\subsection*{Derivative}

$f(x) = sqrt(x)cos^2(x + 1)$


\section*{Midterm}

TODO: add midterm to calendar 2. or 3. of October.

\begin{itemize}
    \item Optimisation (Local / Gloal)
    \item Derivatives
    \item L'Hopital
    \item Word Problem
\end{itemize}

\newpage

\subsection*{L'Hopital}

$lim_{x \to 0} \frac{x-xcos(x)}{x-sin(x)}$

\noindent $f(0) = \frac{0 - 0 \times 1}{0 - 0} = \frac{0}{0} = 0$

\noindent $= 1 - (cos(x)-xsin(x)) = 1 - cos(x) + xsin(x)$

\noindent Use product rule on the counter.

\noindent $h(x) = x - xcos(x)$

\noindent $h'(x) = 1 - [xcos(x)]\frac{d}{dx} = 1 - ((x)(-sin(x)) + (1)(cos(x))) = 1 - (-xsin(x)+cos(x))$

\noindent $g'(x) = [x-sin(x)]\frac{d}{dx} = 1 - cos(x)$

\noindent $f'(x) = \frac{h'(x)}{g'(x)} = \frac{[\frac{x-xcos(x)}{x-sin(x)}]\frac{d}{dx}}{[x-sin(x)\frac{d}{dx}]} = $

\noindent $f'(0) = \frac{1-1 + 0 \times 0}{1 -1} = 0$

\noindent $f''(x) = \frac{h''(x)}{g''(x)} = \frac{xcos(x) + sin(x)}{sin(x)}$

\noindent $h''(x) = [1-cos(x)+xsin(x)]\frac{d}{dx} = 0 - (-sin(x)) + [xsin(x)]\frac{d}{dx}$

$= (x)(cos(x)) + (1)(sin(x)) = xcos(x) + sin(x)$

$= sin(x) +(xcos(x)) + sin(x)$

$g''(x) = [1-cos(x)]\frac{d}{dx} = 0 - (-sin((x)) = sin(x))$ 


\noindent $f''(0) = \frac{0 + 0 + 0}{0} = \frac{0}{0}$

\noindent $f'''(x) = \frac{h'''(x)}{g'''(x)}$

\noindent $h'''(x) = \frac{[sin(x) + xcos(x) +sin(x)]\frac{d}{dx}}{[sin(x)]\frac{d}{dx}} = \frac{cos(x) + cos(x) + sin(x) + cos(x)}{cos(x)}$

\noindent $h'''(0) = \frac{1 + 1 + 1}{1} = \frac{3}{1} = 3$

\newpage

\section*{Word Problem}


Given a parameter $P$, what rectangle has the largest area.

maximize a function = find the derivative

\noindent $P = 2x + 2y$

\noindent $A(x, y) = x \times y$

\noindent $2y = P -2x$

\noindent $y = \frac{P}{2} - \frac{2x}{2} = \frac{P}{2} - x$

\noindent $A(x, y) = x \times (\frac{P}{2} - x) = \frac{P}{2}x - x^2$

\noindent $A(x) = \frac{P}{2}x - x^2$

\noindent $A'(x) = \frac{P}{2} - 2x$

\noindent $\frac{P}{2} - 2x = 0 => x = \frac{P}{4}$

\noindent $2y = P - 2x$

\noindent $y = \frac{P}{2} - \frac{2x}{2} = (\frac{P}{2} - x)$

\noindent $y = (\frac{P}{2} - \frac{P}{4}) = \frac{P}{4}$

\begin{verbatim*}
    P/4
-----------
|         |
|         | p
|         | -
|         | 4
----------|
\end{verbatim*}


$\frac{P}{4} \times \frac{P}{4} = \frac{P2}{16}$

\newpage

\subsection*{Optimisation}

take derivative, set to 0.

\noindent If bounded, check ends.
\noindent $f(x) = xe^-2$ on range $[-\frac{1}{2}, 2]$


\noindent product rule and chain rule

\noindent $f'(x) = [xe^(-x^2)] \frac{d}{dx}$

\noindent $h(x) = x$

\noindent $h'(x) = 1$

\noindent $g(x) =e^-2$

\noindent $g'(x) = [f(g(x))]\frac{d}{dx} = f'(g(x)) \times g'(x)$


\noindent $f(x) = e^x$

\noindent $g(x) = -x^2$

\noindent $f'(x) = e^x$

\noindent $g'(x) = -2x$

\noindent $f'(x) = e^{-x^2} \times (-2x) + e^{-x^2}$

$ = -2x^{2}e^{-x^2} + e^{-x^2} = 0$

\noindent We need $ln$ to get rid of the e.

\noindent $ln(-2x^{2}e^{-x^2} + e^{-x^2}) = ln(0)$

\noindent $2ln(2xe^{-x^2} + e^{-x^2}) = ln(0)$

$2 \times (-x^2) \times (-x^2) \to 2 \times a \times a = 2a^2 \to 2(-x^2)^2 = -2x^4$

this is simpler

$e^{-x^2}(2x^2 + 1)$

$-2x^2 + 1 = 0$


\newpage

\subsection*{Taylor series}
Not on midterm

\newpage

\subsection*{Intergrals}
\begin{tabular}{l l}
    U-substitution       &  Chain Rule \\
    integration by-parts &  Product Rule \\
\end{tabular}
\newline

\begin{tabular}{l l}
    $u = x$             & $dv = e^x dx$ \\
    $\frac{du}{dx} = 1$ & $v = e^x$     \\
    $du = dx$           &               \\
\end{tabular}
\newline

\subsection*{Example 1}

\noindent $\int v dv = uv - \int v dv $

\noindent $\boxed{\int x e^x dx} = x e^x - \int e^x dx$

\noindent $=  xe^x - e^x + C$

\noindent $= e^x (x - 1) + C$

\subsection*{Example 2}

$\int_{0}^{1} x^2 \sqrt(1 - x) dx$ Integration by parts.

\noindent $\int x \sqrt{x} = \int x^\frac{3}{2}$

\noindent $\int_{0}^{1} x^2 \sqrt(1 - x) dx = [-\frac{2}{3}x^2 (1-x)^\frac{3}{2}]_{0}^{1} + \boxed{\frac{2}{3} \int_{0}^{1}ex(1-x)^\frac{3}{2} dx}$

\noindent Plug in the numbers.

\noindent $[-\frac{2}{3} \times \frac{4}{5} \times x(1-x)^\frac{5}{2}]_{0}^{1} + \frac{2}{3} \times \frac{4}{5}$

\noindent $[-\frac{2}{3} \times \frac{4}{5} \times \frac{2}{7} \times (1-x)^\frac{7}{2}]_{0}^{1} = \frac{16}{105}$

\subsection*{Example 3}

U-substitution

\noindent $\int_{0}^{1} x^2 \times \sqrt{1 - x} = - \int_{1}^{0} (1 - u)^2 \times \sqrt{u} du$

\noindent $\boxed{u = 1 - x}$

\noindent $\boxed{dx = -du}$

\noindent $= \int_{0}^{1} (1-2u-u^2)\sqrt{u} du = \int_{0}^{1} (u^\frac{1}{2} - 2u^\frac{3}{2} + u^\frac{5}{2}) du$

\noindent $\int a + b - c = \int a + \int b - \int c$

\noindent $[\frac{2}{3}u^\frac{3}{2}]_{0}^{1} - [-\frac{4}{5}u^\frac{4}{5}]_{0}^{1} + [\frac{2}{7} u^\frac{7}{2}]_{0}^{1} = \frac{70 - 84 + 30}{105} = \boxed{\frac{16}{105}}$
\newline

\noindent How do I know which to use? Do 100 intergrals!

\newpage

\subsection*{Examaple 3}

$\int_{0}^{1} x^2 \sqrt{1 - x} dx$ Integrate by-parts.

\noindent $\boxed{\int u du = uv - \int v du}$

\noindent $u = x^2$ $\frac{du}{dx} = 2x$ $du = 2xdx$

\noindent $dv = \sqrt{1 - x}$ $v = \frac{2(1-x)^\frac{3}{2}}{3}$

\noindent $=(1-x)^\frac{1}{2} = \frac{(1-x)^\frac{3}{2}}{\frac{3}{2}} = \frac{2(1-x)^\frac{3}{2}}{3}$

\noindent $=[\frac{2}{3} x^2 (1-x)^\frac{3}{2}]_{0}^{1} - \frac{2}{3} \int (1 - x)^\frac{3}{2} \sqrt(1-x)$

\noindent $[-\frac{2}{3} x^2 (1 - x)^\frac{3}{2}]_{0}^{1} + \frac{2}{3} \int (1-x)^\frac{3}{2} 2x dx$
\newline

\noindent There might be some errors in this example.

\section*{Linear Algebra}
\subsection*{1.}

$\int_{a}^{b} = F(b) - F(a)$
\newline

\noindent $sin(x)$ and $sin(x+\frac{\pi}{3})$

\noindent $sin(x) = sin(x + \frac{\pi}{3})$

\noindent $sin(x) = sin(x) cos(\frac{\pi}{3}) + cos(x) sin(\frac{pi}{3})$

\noindent $sin(x) = \frac{1}{2} sin(x) + \frac{\sqrt{3}}{2} cos(x)$

\noindent $sin(x) - \frac{1}{2} sin(x) = \frac{\sqrt{3}}{2} cos(x)$

\noindent $\frac{1}{2} sin(x) = \frac{\sqrt{3}}{2} cos(x)$

\noindent $\frac{1}{2} sin(x) = \frac{\sqrt{3}}{2} cos(x)$

\noindent $sin(x) = \sqrt{3} cos(x)$

\noindent $tan(x) = \sqrt{3} cos(x)$

\noindent $tan(x) = \sqrt{3}$

\noindent $k \in \mathds{Z}$
\newline

\noindent $\boxed{x = \frac{\pi}{3} + k \pi}$
\newpage

\noindent $\int_{\frac{-2\pi}{3}}^{\frac{\pi}{3}} sin(x + \frac{\pi}{3}) - sin(x) dx$

\noindent Use angle sum to get:

\noindent $\int_{\frac{-2\pi}{3}}^{\frac{\pi}{3}} cos(x + \frac{\pi}{6}) dx$

\noindent Plug the number in:

\noindent $sin(\frac{\pi}{3} + \frac{\pi}{6}) - sin(\frac{-2\pi}{3} + \frac{\pi}{6}) = 2$

\subsection*{2.}

\begin{tabular}{| l |}
\hline
$A = (1, -5, 5)$ \\
$B = (11, 5, 0)$ \\
$C = (3, 9, 10)$ \\
\hline
\end{tabular}
\newline

\noindent This is a triangle.



\subsubsection*{(a) Side lengths}

\noindent $|| \overline{AC} || = \sqrt{(3-1)^2 + (9+5)^2 + (10-5)^2} = \sqrt{2^2 + 14^2 + 5^2} = 15$

\noindent $|| \overline{AB} || = \sqrt{180}$

\noindent $|| \overline{BC} || = 15$

\noindent The dot product tells you in how much in A directis is B going.

\noindent The cross product gives us a 3D vector.

\noindent Right hand rule tells us if it is going towards us or away from us.

\subsubsection*{(b) Angle from A}

\noindent \begin{tabular}{| l |}
\hline
$A = (a, b, c)$ \\
$B = (d, e, f)$ \\
\hline
\end{tabular}
\newline

\noindent $AB = |A| \times |B| \times cos(p)$

\noindent Dot product $ AB = ad + be + cf$
\noindent

\noindent Vectors:

\noindent $\overline{AC} = ((3-1), (9+5), (10-5)) = (2, 14, 5)$

\noindent $\overline{AB} = ((11-1), (5+5), (0-5)) = (10, 10, -5)$

\noindent $\overline{BC} = ((3-11), (9-5), (10-0)) = (-8, 4, 10)$

\noindent $\overline{AC} \times \overline{AB} = 20 + 140 - 25 = 135$

\noindent Magnitude of the vector: 

\noindent $||\overline{AB}|| = 6\sqrt{5}$

\noindent $||\overline{AC}|| = 15$

\noindent $135 = 6\sqrt{5} \times 15 \times cos(\varphi)$

\noindent $135 = 90 \sqrt{5} \times cos(\varphi)$
\newline

\noindent $cos(\varphi) = \frac{135}{90\sqrt{5}}$
\newline

\noindent $arccos(\frac{135}{90\sqrt{5}})$

\subsubsection*{(c) Area of triangle}

\noindent Cross product $|A \times B| = |A| \times |B| sin(\varphi)$
\newline

\noindent $|\overline{AB} \times \overline{AC}| = |\overline{AB}| \times |\overline{AC}| \times sin(\varphi) = 6\sqrt{5} \times 15 \times sin(0.8) = 149.86$

\subsubsection*{(d) Projection of $\overline{AC}$ to $\overline{AB}$}

$proj_{\overline{AB}}$ $\overline{AC} = \frac{\overline{AB} \times \overline{AC}}{|\overline{AB}|} = \frac{135}{6\sqrt{5}} = \frac{9\sqrt{5}}{2}$

\noindent Dot product: $\overline{AB} \times \overline{AC} = 135$

\noindent $proj_{\overline{AB}}$ $\overline{AC}$ = vector $\times$ scaler = $(2, 14, 5) \times \frac{9\sqrt{5}}{2} = (9\sqrt{5}, 63 \sqrt{5}, \frac{45 \sqrt{5}}{2})$

\subsubsection*{(e) Orthagonal $Proj_{\overline{AB}} \overline{AC}$}

\end{document}
